\newpage
\section{Einsatzgebiete}
Selbstorganisierte Teams kommen in jedem natürlichen sozialen System vor. Prominente Beispiele sind die Armee Cäsars oder Henry Fords Werke. In der jüngsten Zeit ist ein Hauptanwendungsgebiet in der agilen Softwareentwicklung, insbesondere in Scrum-Teams.\cite{Gloger2014}

\section{Rollen}

In einem selbstorganisierten Team bestehen für die Rollen verschiedene Verantwortungen:\cite{Mittal2013}
\begin{description}
	\item[Team-Mitglied] ist ein Person des Projektteams, welches sich in einem Expertengebiet auskennt, dieses im Rahmen des Projekts ausübt und so direkt an der Verwirklichung des Projektziels mitarbeitet.
	
	\item[Scrum-Master] agiert als Trainer und sichert, dass das Team projektspezifische Unterstützung und Training erfährt. Eine weitere Aufgabe ist das Herstellen einer produktiven und angenehmen Arbeitsumgebung für das Team.
	
	\item[Senior-Management] sollte dem Team nicht im Weg stehen. Viel mehr sollte es eine Unterstützende Rolle einnehmen. Auf keinen Fall sollte es die Ziele des Teams beeinflussen - es ist essentiell das Ansätze nicht funktionieren und anschließend andere Ansätze zum Erfolg führen.
	
	\item[Unternehmen] agiert als bereitstellende Instanz für Infrastruktur, Training und antreibende Systeme um die Motivation der Teammitglieder zu fördern oder zu erhalten.
\end{description}

\section{Merkmale}\label{Merkmale}

\begin{quote}
"`Wieso sollten die Mitarbeiter auch selbst denken, wenn sowieso alle Weisheit "`von oben"' herab fällt?"'\cite{Maximini2014}
\end{quote}

Selbstorganisierte Teams sind an den folgenden 6 Merkmalen zu erkennen:\cite{Mittal2013}
\begin{itemize}
    \item Teammitglieder suchen sich selbstständig Aufgaben ohne durch eine Führungsperson darauf hingewiesen worden zu sein. Dies zeigt ein größeres Verständnis der eigenen Verantwortung im Team und erhöhte Motivation.
    \item Teammitglieder verwalten ihre Arbeit als Gruppe. Dies beinhaltet die Arbeitsverteilung, Zeitplanung, Verbesserungen und Ähnliches. Es wird nicht von außen eingegriffen.
    \item Teammitglieder benötigen trotz der zwei vorherigen Punkte Beratung und Fortbildungen, jedoch niemanden, der befiehlt oder kontrolliert.
	\item Teammitglieder kommunizieren im Team sehr ausgiebig und geben sich regelmäßig gegenseitig Feedback. Ebenso gilt das Engagement eher dem Team als dem Scrum-Master.
	\item Teammitglieder verstehen die Anforderungen des Kunden und fragen offen falls sie weitere Informationen benötigen.
    \item Teammitglieder bilden und entwickeln sich ständig weiter und schlagen innovative Ideen und Verbesserungen vor.
\end{itemize}


\section{Voraussetzungen}

\textbf{Voraussetzungen für das System:}\cite{Gloger2014}
\begin{itemize}
	\item Eine für das System existenzielle Mission/Aufgabe
	\item Einen die Systembalance gefährdenden äußeren/inneren Anpassungsdruck
	\item Eine dem System eigene und bewusste/unbewusste Anpassungsbereitschaft
	\item Das Geschehenlassen und Installieren systemeigener Strukturen und Regulationsprozesse
	\item Die Integration der Anpassung in die neu entwickelte Systemrealität
\end{itemize}

\textbf{Voraussetzungen für das System:}\cite{Mittal2013}
\begin{itemize}
	\item Ein Rahmen welcher das System umgibt und definiert. Einfach gesprochen gibt es kein "`selbst"' ohne gleichzeitig "`die Anderen"' zu definieren. In einem Unternehmen könnten das abgeklärte Ziele, eine bestimmte Entwicklungsrichtung oder ähnliches sein.
	\item Mitglieder mit verschiedenen Hintergründen in möglichst vielen Bereichen. Beispiele wären Wissensgebiete, Erfahrungsunterscheide, Bildungsabschlüsse, Alter, Geschlecht oder der kulturelle Hintergrund. Hoch effiziente Teams können die unterschiedlichen Ausprägungen der einzelnen Mitglieder wahrnehmen, akzeptieren und nutzen um bessere, kreativere und einzigartige Ergebnisse zu erzielen.
	\item Kommunikation die Interaktionen sowohl im Team als auch zwischen dem Team und der Umgebung erlaubt. Der Austausch von Informationen, Energie oder Material zwischen verflochtenen Personen oder Abteilungen ist ein kritischer Faktor bei der selbst-Organisierung in einem Unternehmen.
\end{itemize}

\textbf{Essentielle Team-Eigenschaften:}\cite{Mittal2013}
\begin{itemize}
	\item Kompetenz: Team-Mitglieder müssen in Ihren Themengebiete kompetent sein.
	\item Zusammenarbeit: Die Individuen im Team sollten als Team operieren, nicht als einzelne Personen.
	\item Motivation: Team-Motivation ist der Schlüssel zum Erfolg des Teams. Jedes Team-Mitglied sollte auf seine Arbeit fokussiert und an dieser interessiert sein.
	\item Vertrauen und Respekt: Team-Mitglieder müssen sich gegenseitig vertrauen und respektieren. Ebenso sollten sie den Code als gemeinsames Gut betrachten und sich bei Problemen gegenseitig helfen.
	\item Kontinuität: Das Team sollte für eine längere Zeit zusammen bleiben. Wird ein Team regelmäßig in kurzen Abständen verändert, kann dies die Effizienz senken. 
\end{itemize}

\section{Weiterentwicklung}

\begin{quote}
"`[...]a self-organizing team doesn't need "`command and control"', but it does need coaching and mentoring."'\cite{Mittal2013}
\end{quote}

Die Weiterentwicklung eines selbstorganisierten Teams erfolgt über einen 3-stufigen Prozess:\cite{Mittal2013}
\begin{description}
\item[Stufe 1: Training ]Team-Mitglieder müssen trainiert werden, damit diese das gewünschte Kompetenz-Niveau erreichen. Nach dem Ende dieser Phase kann man davon ausgehen, dass das Team imstande wäre, selbstorganisiertes Verhalten an den Tag zu legen.

\item[Stufe 2: Coaching ]Sobald das Team in Schwung kommt, sollte auf einen unterstützenden Führungsstil umgestellt werden. Es sollte auf Probleme geachtet und unterstützt werden. Dieser Aufwand könnte zu Beginn höher sein, als in einem schon länger bestehenden Team. Es sollte ebenfalls darauf geachtet werden, ob das Team in Schwung kommt. Anzeichen hierfür sind unter  \ref{Merkmale} aufgeführt. Ist dies der Fall, kann diese Phase beendet werden. Am Ende dieser Phase kann sich das Team selbst organisieren. Natürlich sollte auch nach dieser Phase auf Probleme geachtet und entsprechend reagiert werden.

\item[Stufe 3: Mentoring ] Sobald das Team sich selbst organisiert, ist es das Ziel diesen Zustand zu persistieren. Hierzu sollten Mentoren dem Team zugewiesen werden, welche das Team weiter fördern. Zusätzlich können Maßnahmen wie eine "`Job Rotation"' genutzt werden, um Monotonie zu verhindern und die Motivation zu fördern.

\end{description}

Natürlich ist ein selbstorganisiertes Team nicht statisch und verändert sich über die Zeit. Immer wenn sich die Zusammensetzung eines Teams verändert, sollten die obigen 3 Schritte wiederholt werden.

\section{Laissez-faire vs. selbstorganisiert}
\begin{quote}
"`[...] interference can only be disruptive of natural harmony and is therefore inefficient and undesirable."'\cite{Gaspard2004}
\end{quote}
\newcommand{\xmark}{\ding{55}}%

\begin{tabular}{lr}
Laissez-faire-Führungsstil &  Selbstorganisierte Teams\\ \hline
Im Mittelpunkt steht das Team & \checkmark \\
Das Teammitglied ist das wichtigste Element & \checkmark \\
Das Team darf Entscheidungen selbstständig Treffen & \checkmark \\
Das Team hat Freiheit in der Arbeitsweise & \checkmark \\
Einwirkungen von außen werden als störend agesehen & \xmark \\
\end{tabular}\\[2ex]
Es gibt durchaus weitreichende Überschneidungen zwischen Laissez-faire und selbstorganisiert, jedoch wird bei einem selbstorganisiertem Team der Team/Scrum-Master stärker als Unterstützung (Training, Coaching, Mentoring) mit einbezogen. 

\newpage
\section{Vor- und Nachteile}

\begin{quote}
"`The best architectures, requirements, and designs emerge from self-organizing teams."'\cite{Beedle2001}
\end{quote}

Selbstorganisierte Teams
\begin{itemize}
	\item[+] erreichen bessere Ergebnisse.
	\item[+] liefern einen höheren Geschäftswert.
	\item[+] sind effektiver in der Zusammenarbeit.
	\item[+] lernen schneller.
	\item[+] sind motivierter und mit mehr Spaß an der Arbeit.
	\item[-] erfordern neue Methoden zum Management des Teams.
\end{itemize}